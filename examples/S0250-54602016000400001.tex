% Generated by jats2tex@0.11.1.0

\newcommand{\journaltitle}{Revista Boliviana de Química}
\newcommand{\abbrevjournaltitle}{Rev. Bol. Quim}
\newcommand{\issn}{0250-5460}
\newcommand{\publishername}{Universidad Mayor de San Andrés}
\title{Chemical study of seedlings of Galipea Longiflora (Evanta); A bio-guided
approach\titlegroup{}}
\author[]{Quiroga Selez, Gabriela}
\author[]{Parra Lizarazu*, Carla}
\author[]{Salamanca Capusiri, Efraín}
\author[]{Flores Quisbert, Esther}
\author[]{Giménez Turba, Alberto}
\affil[]{}
\date{15 12 2016}
\date{15 12 2016}
\def\volume{33}
\def\fpage{134}
\def\lpage{141}
\newcommand{\cclicense}{ $^{\tiny{\textregistered}}$ }
\newcommand{\copyrightyear}{}
\newcommand{\kwdgroup}{seedlings, quinoline alkaloids, leishmanicidal activity,
PAT}0250-5460S0250-546020160004000014
\begingroup
\renewcommand{\section}[1]{\subsection*{#1}}

\begin{abstract}

Abstract The active principies (quinolinic alkaloids) from Galipea longiñora
(Evanta) were isolated and characterized . These alkaloids showed low toxicity
and a good in vitro activity against Leishmania strains. Seedlings of different
ages of this medicinal species were developed. The work has been focused on the
analysis of variations of quinolinic alkaloids present in extracts of the
seedlings and the variation of the antiparasitic activity compared with each
other and with the total alkaloids in adult specimens. Five common major
quinolinic alkaloids were identified in the adult-age tree and in the seedlings.
Fourteen alkaloids were identified in the seedlings, compounds such as
Shikimina, and 2-(3`,4`-metoxiphenylquinoline) appear when the seedlings get
older. As a rule the PAT shows better biological activity than CAT with an IC50
of 13,4<img border=0 width=9 height=8 src="../img/a01\_{}figura06.gif">1,2<img
border=0 width=8 height=10 src="../img/a01\_{}figura07.gif">g/mL on L.
amazonensis and 7,2<img border=0 width=9 height=8
src="../img/a01\_{}figura06.gif">0.55<img border=0 width=8 height=10
src="../img/a01\_{}figura07.gif">g/mL on L. braziliensis . This comparison
between the variation of the antiparasitic activity against the variation of
seedling size and chemical composition of PAT give us a pattern of which
molecules will have influence in the biological activity.

\ifdef{\kwdgroup}{\iflanguage{portuges}{\medskip\noindent\textbf{Palavras-chave:
} \kwdgroup}{}}{}
\ifdef{\kwdgroupen}{\iflanguage{english}{\medskip\noindent\textbf{Keywords:}
\kwdgroupen}{}}{}
\ifdef{\kwdgroupes}{\iflanguage{spanish}{\medskip\noindent\textbf{Palavras
claves:} \kwdgroupes}{}}{}
\ifdef{\kwdgroupfr}{\iflanguage{french}{\medskip\noindent\textbf{Mots clés:}
\kwdgroupfr}{}}{}
\end{abstract}
\endgroup

\begingroup
\renewcommand{\section}[1]{\subsection*{#1}}

\begin{abstract}

Resumen Spanish title: Estudio químico biodirigido de plantines de Galipea
longiñora (Evanta). Los principios activos (alcaloides quinolinicos) de Galipea
longiñora (Evanta) fueron aislados y caracterizados. Estos alcaloides mostraron
baja toxicidad y una buena actividad in vitro frente a cepas de Leishmania. Se
han realizado estudios sobre plantines de diferentes edades de Galipea
longiñora. El trabajo se centró en el análisis de variaciones de la mezcla de
los alcaloides quinolinicos presentes en los extractos de plantines y la
variación de la actividad antiparasitaria; comparados entre sí y con alcaloides
totales de corteza de especímenes adultos. Se identificaron 5 alcaloides
mayoritarios comunes en el árbol en edad adulta y plantines. En los plantines se
identificaron hasta 14 alcaloides, llegando a reportarse compuestos como la:
Shikimanina, y 2-(metoxifeniletilquinolina) cuando los plantines representan
mayor edad. En general, los PAT, presentan mejor actividad biológica con una
IC50 de 13,4±1,2<img border=0 width=8 height=10
src="../img/a01\_{}figura07.gif">g/mL sobre L. amazonensis y 7,2±0.55<img
border=0 width=8 height=10 src="../img/a01\_{}figura07.gif">g/mL sobre L.
braziliensis que el CAT. Esta comparación entre la variación de la actividad
antiparasitaria frente a la variación del tamaño del plantin y a la composición
química de los PAT, nos da una pauta de cuáles serían las moléculas que influyen
en la actividad biológica.

\ifdef{\kwdgroup}{\iflanguage{portuges}{\medskip\noindent\textbf{Palavras-chave:
} \kwdgroup}{}}{}
\ifdef{\kwdgroupen}{\iflanguage{english}{\medskip\noindent\textbf{Keywords:}
\kwdgroupen}{}}{}
\ifdef{\kwdgroupes}{\iflanguage{spanish}{\medskip\noindent\textbf{Palavras
claves:} \kwdgroupes}{}}{}
\ifdef{\kwdgroupfr}{\iflanguage{french}{\medskip\noindent\textbf{Mots clés:}
\kwdgroupfr}{}}{}
\end{abstract}
\endgroup
<p align="right"><font size="2" face="Verdana"><b>ART\&Iacute;CULOS
ORIGINALES</b></font></p> <p align="right">\&nbsp;</p> <p align="center"><font
size="4" face="Verdana"><b>Chemical study of seedlings of Galipea Longiflora
(Evanta); A bio-guided approach</b></font></p> <p align="center">\&nbsp;</p> <p
align="center">\&nbsp;</p> <p align="center"><font face="Verdana"
size="2"><b>Gabriela Quiroga Selez, Carla Parra Lizarazu*, Efraín Salamanca
Capusiri, Esther Flores Quisbert, Alberto Giménez Turba <br> </b></font><font
face="Verdana" size="2">Faculty of Pharmaceutical and Biochemical Sciences,
Biochemical Drug Research Institute IIFB, Universidad Mayor de San Andrés UMSA,
P.O. Box 3239, Av. Saavedra 2224, Phone 5912229021, La Paz, Bolivia, <a
href="mailto:farbio@farbio.edu.bo">farbio@farbio.edu.bo</a></font> <br> <font
face="Verdana" size="2"><b>*Correspond\&Iacute;ng author:</b> <a
href="mailto:mggs88.mggs@gmail.com">mggs88.mggs@gmail.com</a></font></p> <p
align="center">\&nbsp;</p> <p align="center">\&nbsp;</p> <hr> <p
align="justify"><font face="Verdana" size="2"><b>Abstract</b></font></p> <p
align="justify"><font face="Verdana" size="2">The active principies (quinolinic
alkaloids) from <i>Galipea longiñora </i>(Evanta) were isolated and
characterized . These alkaloids showed low toxicity and a good <i>in vitro
</i>activity against <i>Leishmania </i>strains. Seedlings of different ages of
this medicinal species were developed. The work has been focused on the analysis
of variations of quinolinic alkaloids present in extracts of the seedlings and
the variation of the antiparasitic activity compared with each other and with
the total alkaloids in adult specimens.</font></p> <p align="justify"><font
face="Verdana" size="2">Five common major quinolinic alkaloids were identified
in the adult-age tree and in the seedlings. Fourteen alkaloids were identified
in the seedlings, compounds such as Shikimina, and
2-(3`,4`-metoxiphenylquinoline) appear when the seedlings get older.</font></p>
<p align="justify"><font face="Verdana" size="2">As a rule the PAT shows better
biological activity than CAT with an IC50 of 13,4<img
src="img/revistas/rbq/v33n4/a01\_{}figura06.gif" width="9" height="8">1,2<img
src="img/revistas/rbq/v33n4/a01\_{}figura07.gif" width="8" height="10">g/mL on
<i>L. amazonensis </i>and 7,2<img
src="img/revistas/rbq/v33n4/a01\_{}figura06.gif" width="9" height="8">0.55<img
src="img/revistas/rbq/v33n4/a01\_{}figura07.gif" width="8" height="10">g/mL on
<i>L. braziliensis . </i>This comparison between the variation of the
antiparasitic activity against the variation of seedling size and chemical
composition of PAT give us a pattern of which molecules will have influence in
the biological activity.</font></p> <p align="justify"><font face="Verdana"
size="2"><b>Keywords: </b><i>seedlings, quinoline alkaloids, leishmanicidal
activity, PAT. </i></font></p> <hr> <p align="justify"><font face="Verdana"
size="2"><b>Resumen</b></font></p> <p align="justify"><font face="Verdana"
size="2"><b><i>Spanish title: </i></b><i>Estudio químico biodirigido de
plantines de Galipea longiñora (Evanta). </i>Los principios activos (alcaloides
quinolinicos) de <i>Galipea longiñora </i>(Evanta) fueron aislados y
caracterizados. Estos alcaloides mostraron baja toxicidad y una buena actividad
in vitro frente a cepas de <i>Leishmania. </i>Se han realizado estudios sobre
plantines de diferentes edades de <i>Galipea longiñora. </i>El trabajo se centró
en el análisis de variaciones de la mezcla de los alcaloides quinolinicos
presentes en los extractos de plantines y la variación de la actividad
antiparasitaria; comparados entre sí y con alcaloides totales de corteza de
especímenes adultos.</font></p> <p align="justify"><font face="Verdana"
size="2">Se identificaron 5 alcaloides mayoritarios comunes en el árbol en edad
adulta y plantines. En los plantines se identificaron hasta 14 alcaloides,
llegando a reportarse compuestos como la: Shikimanina, y
2-(metoxifeniletilquinolina) cuando los plantines representan mayor
edad.</font></p> <p align="justify"><font face="Verdana" size="2">En general,
los <b>PAT, </b>presentan mejor actividad biológica con una <i>IC50 </i>de
13,4\&plusmn;1,2<img src="img/revistas/rbq/v33n4/a01\_{}figura07.gif" width="8"
height="10">g/mL sobre <i>L. amazonensis y </i>7,2\&plusmn;0.55<img
src="img/revistas/rbq/v33n4/a01\_{}figura07.gif" width="8" height="10">g/mL
sobre <i>L. braziliensis </i>que el <b>CAT. </b>Esta comparación entre la
variación de la actividad antiparasitaria frente a la variación del tamaño del
plantin y a la composición química de los <b>PAT, </b>nos da una pauta de cuáles
serían las moléculas que influyen en la actividad biológica.</font></p> <hr> <p
align="justify">\&nbsp;</p> <p align="justify">\&nbsp;</p> <p
align="justify"><font face="Verdana" size="3"><b>INTRODUCCIÓN</b></font></p> <p
align="justify"><font face="Verdana" size="2">La especie medicinal <i>Galipea
longiñora K. </i>[sinónimo <i>Angostura longiñora </i>(Krause) Kallunki] tiene
diversos nombres comunes, pero es conocida ampliamente por el nombre vernacular
de \&quot;Evanta\&quot;. Pertenece a la familia Rutaceae, es un árbol que crece
hasta 12 metros de altura, presenta hojas trifoliadas alternas o superpuestas
sobre la misma rama con un pecíolo frecuentemente alado y sus flores aparecen en
forma de racimos. En Bolivia se la encuentra en los bosques tropicales de los
últimos contrafuertes andinos en los departamentos de Beniy La Paz
[1].</font></p> <p align="justify"><font face="Verdana" size="2">Esta especie se
encuentra registrada en farmacopeas tradicionales como medicinal en la cura de
problemas digestivos y principalmente de leishmaniasis cutánea (lepra blanca).
Esta sintomatología puede ser interpretada por la medicina occidental, como
antiparasitaria (leishmanicida). Esta planta es usada de manera tradicional por
las etnias Tacana, Mosetén y Chiman. El uso tradicional más frecuente según los
Tacanas es en forma de cocción de la corteza de tronco para el tratamiento de
diarreas causadas por parásitos intestinales y como fortificante para niños y
adultos; para el tratamiento de la leishmaniasis, la corteza fresca o seca es
molida y aplicada directamente en forma de cataplasma sobre las úlceras, dos
veces al día hasta que sane, además de beber la decocción [2].</font></p> <p
align="justify"><font face="Verdana" size="2">Según la Organización Mundial de
la Salud se estima que se presentan cada año 2 millones de casos nuevos en todo
el mundo, de los cuales 1,5 millones corresponden a leishmaniasis cutánea. Sin
embargo, los datos oficiales subestiman la realidad de la afección debido a que
numerosos casos no son diagnosticados, y existen un gran número de casos
asintomáticos. [3].</font></p> <p align="justify"><font face="Verdana"
size="2">La leishmaniasis en Bolivia se encuentra geográficamente distribuida en
el área rural de las zonas tropicales y subtropicales de los departamentos de La
Paz, Beni, Pando, Santa Cruz, Cochabamba, Tanja y Chuquisaca [4]. Entre los años
1985-1991, investigadores del Instituto Boliviano de Biología de Altura (IBBA
UMSA) y del IRD (Francia) estudiaron la actividad antiparasitaria de los
extractos de hojas, raíces y corteza de <i>Galipea longiñora </i>(Evanta),
habiéndose aislado e identificado 15 alcaloides quinoléicos [5,6,7]. Algunos
alcaloides aislados de las hojas mostraron eficacidad y baja toxicidad <i>in
vivo </i>contra leishmaniasis cutánea y fueron objeto de patentes (Chimaninas A,
B, Cy D US4209519/15/04/93) [8, 9,10].</font></p> <p align="justify"><font
face="Verdana" size="2">El uso tradicional de esta planta como antiparasitario
en Bolivia está reconocido dentro de la lista de los recursos naturales
terapéuticos aceptados; las normas para medicamentos naturales lo reconocen como
medicamento seguro, eficaz y de calidad (Ministerio de Salud y Deportes, 2001)
[11]. En la actualidad, el Instituto de Investigaciones Fármaco-Bioquímicas
(IIFB), de la Facultad de Ciencias Farmacéuticas y Bioquímicas de la UMSA, forma
parte de un equipo multidisciplinario que está llevando adelante estudios
clínicos sobre la eficacia de formulaciones farmacéuticas, a base de los
principios activos de Evanta, frente a la leishmaniasis cutánea, así cómo para
el tratamiento de parasitosis intestinales. Se continua con trabajos de
investigación sobre nuevas alternativas para diferentes tratamientos a partir de
extractos vegetales. El objetivo del presente trabajo sobre <i>Galipea
longiñora, </i>se enfoca en el uso de plantines como nueva fuente de principios
activos. De los cientos de plantines producidos, sólo cerca de una decena llegan
hasta adultos.</font></p> <p align="justify"><font face="Verdana" size="2">En
esta investigación, presentamos los resultados de los estudios centrados en el
análisis de variaciones de la mezcla de los alcaloides quinolínicos presentes en
los extractos de plantines y la variación de la actividad antiparasitaria;
comparados entre sí y con alcaloides totales de especímenes de 9 metros,
colectados en la comunidad de Hernández, Provincia Sud Yungas del Departamento
de La Paz. Los extractos obtenidos fueron evaluados por su actividad
leishmanicida sobre formas promastigote, frente a cepas de <i>Leishmania
amazonensis </i>y <i>Leishmania braziliensis.</i></font></p> <p
align="justify">\&nbsp;</p> <p align="justify"><font face="Verdana"
size="3"><b>RESULTADOS</b></font></p> <p align="justify"><font face="Verdana"
size="2">Un total de 113 plantines fueron recolectados y agrupados de acuerdo al
tamaño que estos poseían. Para la obtención de alcaloides totales <b>(PAT)
</b>los plantines de <i>Galipea longiñora </i>(Evanta) fueron utilizados en su
totalidad (hojas, tallo y raíz) estos fueron molidos, pesados y sometidos a
maceración etanólica, para luego obtener los AT por precipitación selectiva
acido - base. Una vez obtenidos los extractos secos se determinó su rendimiento
respecto a la masa de planta seca <b>(<a href="\#t1">Tabla 1</a>),
</b>dividiéndose estos en tres grupos: A (de 25 a 60 cm), B (de 70 a 90 cm) y C
(de 110 a 175 cm) esto de acuerdo al rendimiento de alcaloides
obtenidos.</font></p> <p align="justify"><font face="Verdana" size="2">Como se
puede observar los porcentajes de <b>PAT </b>obtenidos en los tres grupos de
plantines (0.22, 0.59 y 0.93\%) representan menos de la mitad del rendimiento de
alcaloides totales de corteza <b>(CAT) </b>obtenidos de un árbol adulto (2\%).
Sin embargo a su vez se puede observar que mientras mayor edad tenga el plantin
la cantidad de alcaloides obtenidos es mayor, llegando a ser ésta, un factor de
importancia considerable, ya que influye en la cantidad total de principios
activos y la estabilidad presentada por estos.</font></p> <p
align="justify"><font face="Verdana" size="2"><i>Caracterización de alcaloides
de Evanta (Galipea longiñora) por técnicas cromatográficas</i></font></p> <p
align="justify"><font face="Verdana" size="2"><i>Análisis de los alcaloides de
Galipea longiñora por cromatografía en capa fina TLC</i></font></p> <p
align="justify"><font face="Verdana" size="2">Los PAT obtenidos se sometieron a
un análisis por cromatografía en placa fina (TLC), frente al CAT del árbol
adulto, los resultados de este análisis se detallan en la siguiente <a
href="\#f1">figura 1</a>.</font></p> <p align="center"><a name="t1"></a><img
src="img/revistas/rbq/v33n4/a01\_{}figura01.gif" width="502" height="280"></p>
<p align="center"><a name="f1"></a><img
src="img/revistas/rbq/v33n4/a01\_{}figura02.gif" width="570" height="274"></p>
<p align="justify"><font face="Verdana" size="2">El análisis de los extractos de
los PAT y el CAT revelan una banda mayoritaria con un Rf=0, 94 que evidencia la
presencia del alcaloide 2 FQ en todas las muestras (<a href="\#f1">Figura 1</a>.
A). Un sistema de elusión diferente permite evidenciar mejor las diferencias
existentes en la presencia de ciertos componentes que no se observan en el CAT,
así como la variación en abundancia de estos componentes y la presencia de
nuevos componentes que solo se observan en los plantines de mayor edad
(B).</font></p> <p align="justify"><font face="Verdana" size="2"><i>Análisis de
los alcaloides de Galipea longiñora por cromatografía de
gases/masas</i></font></p> <p align="justify"><font face="Verdana" size="2">Se
eligieron plantines de los grupos A, B y C (<a href="\#t1">Tabla 1</a>) para la
identificación de metabolitos. Los resultados de CG/MS obtenidos nos permiten
llevar a cabo el análisis de las relaciones isotópicas de los elementos ligeros
presentes en los alcaloides propios de la Evanta para medir las pequeñas
variaciones en la abundancia isotópica. De esta forma se han podido identificar
y comparar los alcaloides presentes en los plantines comparándolos entre sí y
además con los patrones de análisis obtenidos del fraccionamiento del CAT
etanólico (IIFB-UMSA, La Paz, Bolivia; Universidad de IUBO-Tenerife, España); la
asignación de las estructuras químicas de los alcaloides quinolínicos, se
realizó por comparación del tiempo de retención, abundancia relativa y masa
obtenida.</font></p> <p align="justify"><font face="Verdana" size="2">En los
cromatogramas obtenidos puede observarse las principales diferencias entre los
alcaloides presentes en la corteza del árbol adulto CAT y los presentes en los
plantines (PAT): los cinco alcaloides mayoritarios comunes: la
2-n-propilquinolina (8), 2-n-pentilquinolina (7), 2-n-fenilquinolina (1),
4-metoxi-2-fenilquinolina (2) y la 2-(3',4'-metilendioxifeniletil)-quinolina
(4). Así mismo se observa, en el espectro de los plantines, varios picos
pequeños de diferente abundancia relativa, encontrándose entre la
2-n-pentilquinolina (7) yla 2-n-fenilquinolina (1), haciendo a ésta el área más
variable, respecto a los alcaloides presentes en el árbol adulto, de las cuales,
algunas pertenecen exclusivamente a las hojas. <a href="\#f2">Figura 2</a>
A.</font></p> <p align="center"><a name="f2"></a><img
src="img/revistas/rbq/v33n4/a01\_{}figura03.gif" width="702" height="1001"></p>
<p align="justify"><font face="Verdana" size="2">Con la comparación entre los
alcaloides de los grupos A, B y C (<a href="\#f2">Figura 2</a>. B), podemos
observar que no existen grandes diferencias entre las abundancias relativas de
los alcaloides presentes en los plantines estudiados; la mayor diferencia se
encuentra al comparar la región que abarca los 55 a 78 minutos de retención,
principalmente entre los 60 y 65 minutos; es en esta región donde se encuentran
los alcaloides propios de las hojas i.e. Chimaninas A (9), B (10), C (11) y D
(12).</font></p> <p align="justify"><font face="Verdana" size="2">También se
observa un pico inidentificable a los 62,69<img
src="img/revistas/rbq/v33n4/a01\_{}figura06.gif" width="9" height="8">0.03
minutos de retención (*), debido a que presentó un diferente fraccionamiento y
un peso molecular en los distintos espectros de masas.</font></p> <p
align="justify"><font face="Verdana" size="2">Para analizar si, se presentaba un
patrón en la abundancia relativa de los componentes del concentrado total de
alcaloides en los plantines y el aumento de la edad de estos, se realizó un
análisis comparativo de las relaciones de porcentaje de cada uno de los
alcaloides. (<a href="\#t2">Tabla 2</a>).</font></p> <p align="center"><a
name="t2"></a><img src="img/revistas/rbq/v33n4/a01\_{}figura04.gif" width="551"
height="798"></p> <p align="justify"><font face="Verdana" size="2">Comparando la
abundancia relativa de los alcaloides quinolínicos contenidos en los grupos de
plantines se observa que ésta y por tanto la mezcla de alcaloides varía de
acuerdo al aumento de la edad o tamaño del plantin.</font></p> <p
align="justify"><font face="Verdana" size="2"><i>Actividad biológica in
vitro</i></font></p> <p align="justify"><font face="Verdana" size="2">Se evaluó
la actividad antiparasitaria de los PAT de los grupos A, B y C, sobre
promastigotes de <i>Leishmania, </i>los resultados muestran fluctuaciones
consideradas dependientes de la edad de éstos y las proporciones relativas de
los alcaloides en cada uno de estos plantines. No se observan diferencias
significativas entre ellos, sin embargo, existe una diferencia marcada respecto
al árbol adulto (<a href="\#t3">Tabla 3</a>) presentando los plantines una mejor
actividad biológica frente a las cepas <i>Leishmania </i>estudiadas.</font></p>
<p align="justify"><font face="Verdana" size="2">La mejor actividad la presentó
el plantin de 25 cm, tanto sobre <i>L. amazonensis (CI <sub>50</sub>= </i>11,5)
como sobre <i>L. braziliensis (CI <sub>50</sub>= </i>6,4), siendo estas
actividades menores a las presentadas por el CAT del árbol adulto: <i>L.
amazonensis (CI <sub>50</sub>=19,5) </i>y <i>L. braziliensis (CI <sub>50</sub>=
</i>17,05).</font></p> <p align="center"><a name="t3"></a><img
src="img/revistas/rbq/v33n4/a01\_{}figura05.gif" width="469" height="320"></p>
<p align="justify"><font face="Verdana" size="2">En trabajos anteriores se
reporta que la actividad del CAT puede ser muy variable incluso sobre las mismas
cepas de <i>Leishmania </i>[12], ya que al ser una mezcla de diferentes
alcaloides quinolínicos, éstos influyen de forma independiente en la actividad
total [13].</font></p> <p align="justify">\&nbsp;</p> <p align="justify"><font
face="Verdana" size="3"><b>DISCUSIÓN</b></font></p> <p align="justify"><font
face="Verdana" size="2">Según datos reportados por otros autores, la abundancia
relativa de los alcaloides de <i>Galipea longiñora </i>varía mucho en función
del órgano de la planta utilizada, algunos compuestos se encontrarían en mayor
abundancia (Ej: 2-n-propilquinolina (8)) o exclusivamente en hojas (Chimaninas),
y otros en la raíz (Ej: 2-n-pentilquinolina (7)); la 2-n-fenilquinolina (1) está
presente en todos los órganos de la planta, llegando a representar entre 40 y
70\% de la abundancia relativa total. Luego, la proporción de alcaloides en este
concentrado es muy variable [14, 15], como se observó en el análisis por
cromatografía en capa fina (TLC). También, la aparición de compuestos en los
plantines de mayor tamaño sugiere que las biosíntesis de algunos alcaloides se
darían a edades mayores.</font></p> <p align="justify"><font face="Verdana"
size="2">Cabe recalcar que en el presente estudio de trabajó con el plantin en
su totalidad, es decir, hojas, corteza y raíz; es así que la comparación entre
la variación de la actividad antiparasitaria frente a la variación del tamaño
y/o edad del plantin y a la composición química de los PAT, nos da una pauta de
cuáles son las moléculas que podrían influir de forma directa en la actividad
biológica. Haciendo un análisis comparativo de la abundancia relativa de cada
uno de los alcaloides, el tamaño, y la actividad biológica del plantin, podemos
observar para el grupo A (menor tamaño) que es éste el que mejor actividad
presenta frente a <i>Leishmania. </i>La principal diferencia se observa en la
2-n-pentilquinolina (7) que posee el doble de abundancia relativa (8,5\%) a
diferencia de los otros grupos; también se observa que las Chimaninas D (12) y C
(11) presentan una abundancia ligeramente mayor.</font></p> <p
align="justify"><font face="Verdana" size="2">Si bien los alcaloides
quinolínicos influyen de forma independiente en la actividad total, es decir,
que poseen por si solos diferente actividad biológica, la mezcla de los
diferentes compuestos en proporciones definidas podrá incrementar y/o disminuir
la actividad biológica total. Todos los datos obtenidos nos llevan a concluir
que la edad de la planta tiene una importancia considerable e influye no solo en
la cantidad total de principios activos producidos, sino también en las
proporciones relativas de los componentes de la mezcla activa.</font></p> <p
align="justify">\&nbsp;</p> <p align="justify"><font face="Verdana"
size="3"><b>EXPERIMENTAL</b></font></p> <p align="justify"><font face="Verdana"
size="2"><i>Recolección del material vegetal</i></font></p> <p
align="justify"><font face="Verdana" size="2">Los plantines de Evanta
<i>(Galipea longiñorá) </i>fueron recolectados en el mes de septiembre de 2012
en la comunidad de Hernández (S: 15\&deg;21' 527\&quot; y O:
067\&deg;20'549\&quot;), provincia Sud yungas del departamento de La Paz,
Bolivia. La identificación taxonómica se realizó mediante comparación, con
muestras Vaucher (AS49 mayo de 1996 y SD17 septiembre de 1994) depositadas en el
Herbario Nacional de Bolivia. Los plantines fueron secados al abrigo de la luz
del sol hasta obtenerse pesos constantes.</font></p> <p align="justify"><font
face="Verdana" size="2"><i>Obtención del concentrado de alcaloides totales de
plantin (PAT)</i></font></p> <p align="justify"><font face="Verdana"
size="2">Hojas, tallos y raíces de plantines de <i>Galipea longiñorá
</i>(Evanta) fueron molidos y macerados en un percolador con alcohol etílico en
una relación 1:5 (p/v) durante 24 horas. Después de la filtración, el extracto
etanólico crudo fue concentrado a presión reducida hasta sequedad. El extracto
crudo se disolvió en HCL, 1N y luego se alcalinizó con NaOH 2N hasta pH básico
(papel pH). Se extrajo contra CH2CI2 v/v 1:1 en embudo de separación liq-liq. Se
repitió la operación tres veces. Después de la separación, NaCl anhidro fue
adicionado a la fase orgánica acumulada con agitación magnética por 15 minutos
para adsorber trazas de agua. Se filtró el NaCl. El filtrado (fase orgánica
libre de trazas de agua) fue evaporada a sequedad a presión reducida para dar el
PAT (rendimiento.....\%).</font></p> <p align="justify"><font face="Verdana"
size="2"><i>Análisis de los extractos obtenidos por cromatografía en capa
fina</i></font></p> <p align="justify"><font face="Verdana" size="2">Los
extractos obtenidos de los plantines fueron sometidos a cromatografía en capa
fina (TLC) utilizando placas de silicagel 60 F254 (Merck) con base de aluminio,
se empleó un sistema de elusión compuesto por CH2CI2-: MeOH (5\%) y Éter:
Acetato de Etilo (10\%) , las placas fueron reveladas bajo luz UV a 254 nm y 365
nm. La corrida de los extractos se realizó junto a un patrón de 2FQ para la
comparación de Rf's.</font></p> <p align="justify"><font face="Verdana"
size="2"><i>Cromatografía de gases acoplada a espectrómetro de masas
(CG/MS)</i></font></p> <p align="justify"><font face="Verdana" size="2">El
extracto PAT fue analizado en un Cromatógrafo de Gases acoplado a Espectrómetro
de Masas SHIMADZU, sobre una columna capilar CARBOWAX (0.25um x 30m), bajo las
siguientes condiciones: T\&deg; del horno: 40\&deg;C por 5min y ratio
2\&deg;C/min, T\&deg; del horno: 220\&deg;C por 25 min, Inyector (250\&deg;C),
gas carrier H2 (modo flujo 1.78mL/min), detector de masas 2 min - 120 min (start
m/z 15, End m/z 400), detector de ionización a flama (FID)
(250\&deg;C).</font></p> <p align="justify"><font face="Verdana"
size="2"><i>Evaluaciones antiparasitarias in vitro</i></font></p> <p
align="justify"><font face="Verdana" size="2">La actividad de las muestras fue
evaluada a través del método colorimétrico XTT (X-4251 Sigma-Aldrich) sobre
cultivos <i>in vitro </i>del estadio promastigote de <i>Leishmania: </i>complejo
<i>Leishmania amazonensis </i>(cepa Lma), complejo <i>Leishmania braziliensis
</i>(cepa M2904). En cada modelo la CI50 del extracto fue determinada mediante
análisis de regresión lineal (porcentaje de inhibición <i>vs. </i>logaritmo de
la concentración del extracto) [15]. Ambas pruebas se realizaron en el
Laboratorio de Quimioterapia Experimental del Instituto de Investigaciones
Fármaco Bioquímicas de la Facultad de Ciencias Farmacéuticas y Bioquímicas de la
Universidad Mayor de San Andrés (La Paz - Bolivia).</font></p> <p
align="justify">\&nbsp;</p> <p align="justify"><font face="Verdana"
size="3"><b>REFERENCIAS</b></font></p> <!-- ref --><p align="justify"><font
face="Verdana" size="2">1.\&nbsp; \&nbsp; \&nbsp; Giménez A, Ávila JA, Ruiz G,
Paz M, Udaeta E, Ticona JC, Salamanca E, Paredes C, Rodríguez N, Quints K,
Feraudy C,</font> <font face="Verdana" size="2">Gutiérrez I, Chuqui R, Quenevo
C, Dalence MF, Bascope M. <b>2005</b>, Estudios químicos, biológicos y
farmacológicos de <i>Galipea longiñorá </i>Krause. <i>Rev Bol Quím.; 22(1),
</i>94-107. \&nbsp;\&nbsp;\&nbsp;\&nbsp;\&nbsp;\&nbsp;\&nbsp;\&nbsp;[\&\#160;<a
href="javascript:void(0);" onclick="javascript: window.open('/scielo.php?script=
sci\_{}nlinks\&ref=284943\&pid=S0250-5460201600040000100001\&lng=','','width=640
,height=500,resizable=yes,scrollbars=1,menubar=yes,');">Links</a>\&\#160;]<!--
end-ref --></font></p> <!-- ref --><p align="justify"><font face="Verdana"
size="2">2.\&nbsp; \&nbsp; \&nbsp; UMSA: IIFB-IIQ-IBBA, FONAMA-EIA. <b>1999,
</b>Tacana: conozcan nuestros árboles, nuestras hierbas. La Paz: Centro de
información para el desarrollo.
\&nbsp;\&nbsp;\&nbsp;\&nbsp;\&nbsp;\&nbsp;\&nbsp;\&nbsp;[\&\#160;<a
href="javascript:void(0);" onclick="javascript: window.open('/scielo.php?script=
sci\_{}nlinks\&ref=284945\&pid=S0250-5460201600040000100002\&lng=','','width=640
,height=500,resizable=yes,scrollbars=1,menubar=yes,');">Links</a>\&\#160;]<!--
end-ref --></font></p> <!-- ref --><p align="justify"><font face="Verdana"
size="2">3.\&nbsp; \&nbsp; \&nbsp; Ministerio de Salud y Deportes. Situación
general y tendencias. Respuesta del sistema de salud OPS/OMS representación
Bolivia; <b>2004, </b>Noviembre [acceso 20 de junio de 2014]. Disponible
</font><font face="Verdana" size="2" color="\#000000">en:</font><font
face="Verdana" size="2"><a href="http://saludpublica,bvsp,org,bo/ass/analisissit
uacion/bolivia/perfil-bolivia-2004.pdf" target="\_{}blank">http://saludpublica,b
vsp,org,bo/ass/analisissituacion/bolivia/perfil-bolivia-2004.pdf</a></font>\&nbs
p;\&nbsp;\&nbsp;\&nbsp;\&nbsp;\&nbsp;\&nbsp;\&nbsp;[\&\#160;<a
href="javascript:void(0);" onclick="javascript: window.open('/scielo.php?script=
sci\_{}nlinks\&ref=284947\&pid=S0250-5460201600040000100003\&lng=','','width=640
,height=500,resizable=yes,scrollbars=1,menubar=yes,');">Links</a>\&\#160;]</p><!
-- end-ref --><!-- ref --><p align="justify"><font face="Verdana"
size="2">4.\&nbsp; \&nbsp; \&nbsp; Mollinedo S. coordinador. <b>2007,
</b>Leishmaniasis guía operativa para el control en Bolivia. 1<sup>ed</sup>. La
Paz: Programa Nacional de Control de las Leishmaniasis.
\&nbsp;\&nbsp;\&nbsp;\&nbsp;\&nbsp;\&nbsp;\&nbsp;\&nbsp;[\&\#160;<a
href="javascript:void(0);" onclick="javascript: window.open('/scielo.php?script=
sci\_{}nlinks\&ref=284948\&pid=S0250-5460201600040000100004\&lng=','','width=640
,height=500,resizable=yes,scrollbars=1,menubar=yes,');">Links</a>\&\#160;]<!--
end-ref --></font></p> <!-- ref --><p align="justify"><font face="Verdana"
size="2">5.\&nbsp; \&nbsp; \&nbsp; Fournet A, Vagneur B, Richomme P, Bruneton J.
<b>1989, </b>Aryl-2 et alkyl-2 quinoléines nouvelles isolées d'une Rutacée
bolivienne: <i>Galipea ¡ongiflora.. Can J Chem.; 67, </i>2116-2118.
\&nbsp;\&nbsp;\&nbsp;\&nbsp;\&nbsp;\&nbsp;\&nbsp;\&nbsp;[\&\#160;<a
href="javascript:void(0);" onclick="javascript: window.open('/scielo.php?script=
sci\_{}nlinks\&ref=284950\&pid=S0250-5460201600040000100005\&lng=','','width=640
,height=500,resizable=yes,scrollbars=1,menubar=yes,');">Links</a>\&\#160;]<!--
end-ref --></font></p> <!-- ref --><p align="justify"><font face="Verdana"
size="2">6.\&nbsp; \&nbsp; \&nbsp; Fournet A, Hocquemiller R, Roblot F, Cave A,
Richomme P, Bruneton J. <b>1993 </b>Les Chimanines, Nouvelles Quinoléines
Substituees En 2, Isolees D'une Plante Bolivienne Antiparasitaire: <i>Galipea
longíflora. J Nat Prod.; 56 </i>(9), 1547-1552</font>\&nbsp;\&nbsp;\&nbsp;\&nbsp
;\&nbsp;\&nbsp;\&nbsp;\&nbsp;[\&\#160;<a href="javascript:void(0);"
onclick="javascript: window.open('/scielo.php?script=sci\_{}nlinks\&ref=284952\&
pid=S0250-5460201600040000100006\&lng=','','width=640,height=500,resizable=yes,s
crollbars=1,menubar=yes,');">Links</a>\&\#160;]</p><!-- end-ref --><!-- ref
--><p align="justify"><font face="Verdana" size="2">7.\&nbsp; \&nbsp; \&nbsp;
Muñoz MH, Mayrarque J, Fournet A, Gantier JC, Hocquemiller R, Moskowitz H.
<b>1994, </b>Synthesis of an antileishmanial alkaloid isolated from <i>Galipea
longíflora </i>and of related compounds. <i>Chem Pharm Bull; 42 </i>(9),
1914-1916. \&nbsp;\&nbsp;\&nbsp;\&nbsp;\&nbsp;\&nbsp;\&nbsp;\&nbsp;[\&\#160;<a
href="javascript:void(0);" onclick="javascript: window.open('/scielo.php?script=
sci\_{}nlinks\&ref=284953\&pid=S0250-5460201600040000100007\&lng=','','width=640
,height=500,resizable=yes,scrollbars=1,menubar=yes,');">Links</a>\&\#160;]<!--
end-ref --></font></p> <!-- ref --><p align="justify"><font face="Verdana"
size="2">8.\&nbsp; \&nbsp; \&nbsp; Fournet A, Gantier JC, Gautheret A, Leysalles
L, Muñoz MH, Mayrargue J, Moskowitz H, Cavé A, Hocquemiller R. <b>1994, </b>The
activity of 2-substituted quinoline alkaloids in BALB/c mice infected with
<i>Leishmania Donovan. Antimicroh Agents Chemother. 33: </i>537-544.
\&nbsp;\&nbsp;\&nbsp;\&nbsp;\&nbsp;\&nbsp;\&nbsp;\&nbsp;[\&\#160;<a
href="javascript:void(0);" onclick="javascript: window.open('/scielo.php?script=
sci\_{}nlinks\&ref=284955\&pid=S0250-5460201600040000100008\&lng=','','width=640
,height=500,resizable=yes,scrollbars=1,menubar=yes,');">Links</a>\&\#160;]<!--
end-ref --></font></p> <!-- ref --><p align="justify"><font face="Verdana"
size="2">9.\&nbsp; \&nbsp; \&nbsp; Fournet A, Ferreira ME, Rojas A, Torres S,
Fuentes S, Nakayama H, Schinini A, Hocquemiller R. <b>1996, </b>In Vivo Efficacy
of Oral and Intralesional Administration of 2-Substituted Quinolines in
Experimental Treatment of New World Cutaneous Leishmaniasis Caused by
<i>Leishmania amazonensis. Antimicroh Agents Chemother, 40(11), </i>2447-2451.
\&nbsp;\&nbsp;\&nbsp;\&nbsp;\&nbsp;\&nbsp;\&nbsp;\&nbsp;[\&\#160;<a
href="javascript:void(0);" onclick="javascript: window.open('/scielo.php?script=
sci\_{}nlinks\&ref=284957\&pid=S0250-5460201600040000100009\&lng=','','width=640
,height=500,resizable=yes,scrollbars=1,menubar=yes,');">Links</a>\&\#160;]<!--
end-ref --></font></p> <!-- ref --><p align="justify"><font face="Verdana"
size="2">10.\&nbsp; \&nbsp; Fournet A, Angelo A, Muñoz V, Hocquemiller R, Cavé
A, Bruneton J. . <b>1993, </b>2-Substituted Quinoline Alkaloids as Potencial
Antileishmanial Drugs. <i>Antimicroh Agents Chemother, 37 (4), </i>859-863.
\&nbsp;\&nbsp;\&nbsp;\&nbsp;\&nbsp;\&nbsp;\&nbsp;\&nbsp;[\&\#160;<a
href="javascript:void(0);" onclick="javascript: window.open('/scielo.php?script=
sci\_{}nlinks\&ref=284959\&pid=S0250-5460201600040000100010\&lng=','','width=640
,height=500,resizable=yes,scrollbars=1,menubar=yes,');">Links</a>\&\#160;]<!--
end-ref --></font></p> <!-- ref --><p align="justify"><font face="Verdana"
size="2">11.\&nbsp; \&nbsp; Estado plurinacional de Bolivia. Ministerio de salud
y deportes. <b>2001, </b>Normas para medicamentos naturales, tradicionales y
homeopáticos.
\&nbsp;\&nbsp;\&nbsp;\&nbsp;\&nbsp;\&nbsp;\&nbsp;\&nbsp;[\&\#160;<a
href="javascript:void(0);" onclick="javascript: window.open('/scielo.php?script=
sci\_{}nlinks\&ref=284961\&pid=S0250-5460201600040000100011\&lng=','','width=640
,height=500,resizable=yes,scrollbars=1,menubar=yes,');">Links</a>\&\#160;]<!--
end-ref --></font></p> <!-- ref --><p align="justify"><font face="Verdana"
size="2">12.\&nbsp; \&nbsp; Salamanca E. <b>2008, </b>Actividad antiparasitaria
múltiple de alcaloides totales de corteza de <i>Galipea longíflora </i>krause
kallunki (Evanta) [tesis de Maestría]. La Paz: Instituto de Investigaciones
Fármaco Bioquímicas. Universidad Mayor de San Andrés;
\&nbsp;\&nbsp;\&nbsp;\&nbsp;\&nbsp;\&nbsp;\&nbsp;\&nbsp;[\&\#160;<a
href="javascript:void(0);" onclick="javascript: window.open('/scielo.php?script=
sci\_{}nlinks\&ref=284963\&pid=S0250-5460201600040000100012\&lng=','','width=640
,height=500,resizable=yes,scrollbars=1,menubar=yes,');">Links</a>\&\#160;]<!--
end-ref --></font></p> <!-- ref --><p align="justify"><font face="Verdana"
size="2">13.\&nbsp; \&nbsp; Giménez A., Ávila JA., Ruiz G., Paz M., Udaeta E.
<b>2005, </b>Estudios químicos, biológicos y farmacológicos de
<i>Galipealongiflora Krause. Rev Bol Quim. 22 </i>(1), 94-107.
\&nbsp;\&nbsp;\&nbsp;\&nbsp;\&nbsp;\&nbsp;\&nbsp;\&nbsp;[\&\#160;<a
href="javascript:void(0);" onclick="javascript: window.open('/scielo.php?script=
sci\_{}nlinks\&ref=284965\&pid=S0250-5460201600040000100013\&lng=','','width=640
,height=500,resizable=yes,scrollbars=1,menubar=yes,');">Links</a>\&\#160;]<!--
end-ref --></font></p> <!-- ref --><p align="justify"><font face="Verdana"
size="2">14.\&nbsp; \&nbsp; Espinoza B., <b>2011, </b>Estudio de Plantas
Antiparasitarias de Farmacopeas Tradicionales de Bolivia <i>\&quot;Galipea
longíflora Krause y Píper hispidum Swartz. </i>[Tesis de post-grado].
Universidad Mayor de San Andrés. La Paz-Bolivia.
\&nbsp;\&nbsp;\&nbsp;\&nbsp;\&nbsp;\&nbsp;\&nbsp;\&nbsp;[\&\#160;<a
href="javascript:void(0);" onclick="javascript: window.open('/scielo.php?script=
sci\_{}nlinks\&ref=284967\&pid=S0250-5460201600040000100014\&lng=','','width=640
,height=500,resizable=yes,scrollbars=1,menubar=yes,');">Links</a>\&\#160;]<!--
end-ref --></font></p> <!-- ref --><p align="justify"><font face="Verdana"
size="2">15.\&nbsp; \&nbsp; Llanos Medina F., Espinoza Cruz B., Salamanca
Capusiri E., Chuqui R., Flores Quisbert N., Giménez Turba A. <b>2009,
</b>Extracción acuosa de corteza de Galipea longiflora y su actividad
leishmanicida. <i>Biofarbo 17 </i>(2), 32-38.
\&nbsp;\&nbsp;\&nbsp;\&nbsp;\&nbsp;\&nbsp;\&nbsp;\&nbsp;[\&\#160;<a
href="javascript:void(0);" onclick="javascript: window.open('/scielo.php?script=
sci\_{}nlinks\&ref=284969\&pid=S0250-5460201600040000100015\&lng=','','width=640
,height=500,resizable=yes,scrollbars=1,menubar=yes,');">Links</a>\&\#160;]<!--
end-ref --></font></p> <!-- ref --><p align="justify"><font face="Verdana"
size="2">16.\&nbsp; \&nbsp; Salamanca E, Ruiz G, Ticona JC, Giménez A. <b>2008,
</b>Método colorimétrico - XTT: como evaluación de alto rendimiento de
sustancias con actividad leishmanicida. <i>Biofarbo 16, </i>21-27.
\&nbsp;\&nbsp;\&nbsp;\&nbsp;\&nbsp;\&nbsp;\&nbsp;\&nbsp;[\&\#160;<a
href="javascript:void(0);" onclick="javascript: window.open('/scielo.php?script=
sci\_{}nlinks\&ref=284971\&pid=S0250-5460201600040000100016\&lng=','','width=640
,height=500,resizable=yes,scrollbars=1,menubar=yes,');">Links</a>\&\#160;]<!--
end-ref --></font></p> <p align="justify">\&nbsp;</p>
\section*{}
\begin{itemize}

\item[1.]

\item[2.]

\item[3.]

\item[4.]

\item[5.]

\item[6.]

\item[7.]

\item[8.]

\item[9.]

\item[10.]

\item[11.]

\item[12.]

\item[13.]

\item[14.]

\item[15.]

\item[16.]

\end{itemize}
