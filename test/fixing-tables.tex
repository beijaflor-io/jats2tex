% Generated by jats2tex@0.11.1.0

\begingroup
%\rowcolors{1}{verde}{verdelimao}
\begin{table}
\caption{Table 1}
\begin{adjustbox}{width=1\textwidth,center}
\begin{threeparttable}
% COMMENT: rever contagem de colunas! TEM QUE SER 7!
\begin{table}
\begin{xtabular}{ l l l l l l l l l l l l l l l l }\hline

Table 1Access to medicines for non-communicable diseases in adults and seniors
(≥ 20 years), according to socioeconomic, demographic, and health-related
variables. PNAUM, Brazil, 2014.Variable & Prevalence of access to medicines for
NCD\\ \hline
\\ \hline
Full & Partial & Null & p & \textsuperscript{b}
\\ \hline
\\ \hline
\% & \textsuperscript{a}
& 95\%CI & \% & \textsuperscript{a}
& 95\%CI & \% & \textsuperscript{a}
& 95\%CI\\ \hline
Sex
& \multicolumn{6}{l}{}
& 0.025
\\ \hline

Male
& 95.6
& 94.3–96.6
& 3.9
& 2.9–5.2
& 0.5
& 0.3–0.9
& -
\\ \hline

Female
& 93.6
& 92.4–94.6
& 5.8
& 4.9–7.0
& 0.6
& 0.4–0.8
& -
\\ \hline

\multicolumn{7}{l}{Age group (years)}
& < 0.001
\\ \hline

20-39
& 91.2
& 87.5–93.8
& 7.5
& 4.9–11.3
& 1.3
& 0.8–2.3
& -
\\ \hline

40-59
& 93.5
& 92.2–94.6
& 5.8
& 4.7–7.0
& 0.7
& 0.5–1.1
& -
\\ \hline

≥ 60
& 96.2
& 95.3 –96.9
& 3.7
& 3.0–4.6
& 0.1
& 0.04–0.1
& -
\\ \hline

Educational level\textsuperscript{c}

& \multicolumn{2}{l}{}
& \multicolumn{2}{l}{}
& \multicolumn{2}{l}{}
& 0.032
\\ \hline

0-4
& 95.1
& 93.9–96.0
& 4.4
& 3.5–5.5
& 0.5
& 0.3–0.9
& -
\\ \hline

5-8
& 95.1
& 93.7–96.3
& 4.5
& 3.4–5.9
& 0.4
& 0.2–0.7
& -
\\ \hline

9-11
& 92.6
& 90.7–94.2
& 6.8
& 5.2–8.7
& 0.6
& 0.3–1.1
& -
\\ \hline

≥ 12
& 95.1
& 92.7–96.7
& 4.3
& 2.7–6.6
& 0.7
& 0.3–1.6
& -
\\ \hline

Region
& \multicolumn{6}{l}{}
& < 0.001
\\ \hline

North
& 93.6
& 91.1–95.4
& 4.7
& 3.2–6.9
& 1.7
& 0.9–3.2
& -
\\ \hline

Northeast
& 92.0
& 90.2–93.5
& 6.8
& 5.4–8.6
& 1.2
& 0.8–1.9
& -
\\ \hline

Southeast
& 94.9
& 93.4–96.1
& 4.8
& 3.7–6.4
& 0.3
& 0.1–0.6
& -
\\ \hline

South
& 95.8
& 94.4–96.9
& 3.9
& 2.9–5.2
& 0.3
& 0.1–0.8
& -
\\ \hline

Midwest
& 93.9
& 92.2–95.2
& 5.8
& 4.5–7.3
& 0.3
& 0.1–0.8
& -
\\ \hline

\multicolumn{7}{l}{CCEB\textsuperscript{d}
}
& 0.004
\\ \hline

A/B
& 96.4
& 94.7–97.5
& 3.2
& 2.1–4.9
& 0.4
& 0.1–1.0
& -
\\ \hline

C
& 94.1
& 93.0–95.1
& 5.5
& 4.5–6.6
& 0.5
& 0.3–0.7
& -
\\ \hline

D
& 92.8
& 90.5–94.6
& 6.1
& 4.5–8.3
& 1.1
& 0.6–2.0
& -
\\ \hline

E
& 90.8
& 85.3–94.3
& 8.7
& 5.2–14.2
& 0.5
& 0.2–1.4
& -
\\ \hline

\multicolumn{7}{l}{Number of NCD}
& < 0.001
\\ \hline

1
& 96.7
& 95.7–97.5
& 2,5
& 1.8–3.5
& 0,7
& 0.5–1.1
& -
\\ \hline

2
& 93.5
& 91.7–94.9
& 5,9
& 4.6–7.6
& 0,6
& 0.3–1.1
& -
\\ \hline

≥ 3
& 91.3
& 89.7–92.8
& 8,5
& 7.1–10.1
& 0,2
& 0.1–0.4
& -
\\ \hline

\multicolumn{7}{l}{Number of medicines needed\textsuperscript{e}
}
& < 0.001
\\ \hline

1
& 97.0
& 95.7–98.0
& 2.7
& 1.8–4.0
& 0.3
& 0.1–0.6
& -
\\ \hline

2
& 94.4
& 92.7–95.7
& 5.5
& 4.2–7.2
& 0.2
& 0.04–0.7
& -
\\ \hline

3-4
& 94.2
& 92.4–95.5
& 5.8
& 4.5–7.6
& \multicolumn{2}{l}{-}
&
\\ \hline

≥ 5
& 91.1
& 88.4–93.2
& 8.9
& 6.8–11.6
& \multicolumn{2}{l}{-}
&
\\ \hline

\multicolumn{7}{l}{Self-assessment of health}
& < 0.001
\\ \hline

Very good/Good
& 96.5
& 95.5–97.3
& 3.1
& 2.3–4.1
& 0.4
& 0.2–0.7
& -
\\ \hline

Regular
& 93.2
& 91.8–94.3
& 6.2
& 5.1–7.6
& 0.6
& 0.4–1.0
& -
\\ \hline

Bad/Very bad
& 86.8
& 83.6–89.4
& 12.4
& 9.8–15.5
& 0.8
& 0.4–1.8
& -
\\ \hline

All
& 94.3
& 93.4–95.1
& 5.2
& 4.4–6.0
& 0.5
& 0.4–0.7
& -
\\ \hline

\end{xtabular}
\begin{tablenotes}

NCD: non-communicable diseases

\textsuperscript{a}
Percentage adjusted by sample weights and by post-stratification according to
age and sex.

\textsuperscript{b}
Pearson’s Chi-squared test.

\textsuperscript{c}
In completed grades in school.

\textsuperscript{d}
According to the 2013 Brazil Economic Classification Criterion (CCEB 2013) of
the Brazilian Association of Research Companies (ABEP). Available from:
\href{http://www.abep.org}

\textsuperscript{e}
Medicines prescribed by the physician.

%% COMMENT: AQUI VÃO AS NOTAS DE TABELA. EX.
% \item NCD: non-communicable diseases
% \item[a] Percentage adjusted b
% \item[b] Pearson’s Chi-squared test.
% \item[c] In completed grades in school.
% \item[d] According te
% \item[e] Medicines prescribed by the physician.
\end{tablenotes}
\end{threeparttable}
\end{adjustbox}
\end{table}
