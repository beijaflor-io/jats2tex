% Generated by jats2tex@0.11.1.0
\section{BRAZILIAN UNIFIED HEALTH SYSTEM QUALITY AND PATIENT SAFETY}

PARÁGRAFO INICIAL.

\begin{itemize}
\item

Elimina a necessidade de utilização de sistemas de envio de documentos internos
às
instituições e destas para o sistema nacional de notificações, o que reduz o
tempo da
chegada da informação e reduz gastos com o envio delas;

\item

pode-se eliminar a possibilidade de extravio e perda de informações,
especialmente se
estas forem preservadas em bancos de dados redundantes e cópias de segurança,
sem
necessidade de espaço para arquivo físico, além de permitir o manuseio ágil das
informações e a análise de indicadores de gestão;

\item

é possível requerer mais informações sobre os incidentes sem dificultar a coleta
dos
dados, melhorando a qualidade das informações e ampliando a participação dos
profissionais de saúde, o que não é possível com o sistema manuscrito.

\end{itemize}

PARÁGRAFO FINAL. Quanto aos aspectos sociais da sustentabilidade, reduzir o
tempo gasto para o envio do
relato aumenta a participação dos profissionais de saúde com as notificações,
bem como sua
disponibilidade junto aos pacientes possibilitam o envolvimento do paciente e
seus
cuidadores no processo de monitorização de riscos e incidentes em saúde. Em uma
política
nacional, esses atores são fontes importantes de notificação voluntária. Com o
sistema
informatizado de notificações em plataforma web, é possível que qualquer pessoa
com acesso à
internet faça uma notificação.
